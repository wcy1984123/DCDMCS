\defmodule {JohnsonSLDist}

Extends the class \class{ContinuousDistribution} for
the {\em Johnson $S_L$\/} distribution\latex{ (see \cite{tJOH49a,tJOH95a})}.
It  has shape parameters $\gamma$ and $\delta > 0$, location parameter
$\xi$, and scale parameter $\lambda > 0$.
Denoting $t=(x-\xi)/\lambda$ and $z = \gamma + \delta\ln(t)$,
the distribution has density
%
\eq
 f(x) =\html{\delta e^{-z^2/2}/(}\latex{\frac {\delta e^{-z^2/2}}}
 {\lambda t \sqrt{2\pi}} \html{)},
 \qquad  \mbox{for } \xi < x < \infty,
\endeq
%
and distribution function
\eq
 F(x) = \Phi(z), \qquad  \mbox{for } \xi < x < \infty,
\endeq
where $\Phi$ is the standard normal distribution function.
The inverse distribution function is
\eq
 F^{-1} (u) = \xi + \lambda e^{v(u)},  \qquad\mbox{for }  0 \le u \le 1,
\endeq
where
\eq
  v(u) = [\Phi^{-1}(u) - \gamma]/\delta.
\endeq

Without loss of generality, one may choose $\gamma = 0$ or $\lambda=1$.
% This class relies on the methods  \clsexternalmethod{}{NormalDist}{cdf01}{} and
%    \clsexternalmethod{}{NormalDist}{inverseF01}{} of \class{NormalDist} to
%   approximate $\Phi$ and $\Phi^{-1}$.


\bigskip\hrule

%%%%%%%%%%%%%%%%%%%%%%%%%%%%%%%%%%%%%%%%
\begin{code}
\begin{hide}
/*
 * Class:        JohnsonSLDist
 * Description:  Johnson S_L distribution
 * Environment:  Java
 * Software:     SSJ
 * Copyright (C) 2001  Pierre L'Ecuyer and Universite de Montreal
 * Organization: DIRO, Universite de Montreal
 * @author       Richard Simard
 * @since        July 2012

 * SSJ is free software: you can redistribute it and/or modify it under
 * the terms of the GNU General Public License (GPL) as published by the
 * Free Software Foundation, either version 3 of the License, or
 * any later version.

 * SSJ is distributed in the hope that it will be useful,
 * but WITHOUT ANY WARRANTY; without even the implied warranty of
 * MERCHANTABILITY or FITNESS FOR A PARTICULAR PURPOSE.  See the
 * GNU General Public License for more details.

 * A copy of the GNU General Public License is available at
   <a href="http://www.gnu.org/licenses">GPL licence site</a>.
 */
\end{hide}
package umontreal.iro.lecuyer.probdist;\begin{hide}
import umontreal.iro.lecuyer.util.*;
import umontreal.iro.lecuyer.functions.MathFunction;
import optimization.*;
\end{hide}

public class JohnsonSLDist extends JohnsonSystem\begin{hide} {

   private static class Function implements MathFunction
   {
      // To find value of t > 0 in (Johnson 1949, eq. 16)
      protected double a;

      public Function (double sb1) {
         a = sb1;
      }

      public double evaluate (double t) {
         return (t*t*t - 3*t - a);
      }
   }


   private static double[] initPar (double[] x, int n, double xmin)
   {
      // Use moments to estimate initial values of params as input to MLE
      // (Johnson 1949, Biometrika 36, p. 149)
      int j;
      double sum = 0.0;
      for (j = 0; j < n; j++) {
         sum += x[j];
      }
      double mean = sum / n;

      double v;
      double sum3 = 0.0;
      sum = 0;
      for (j = 0; j < n; j++) {
         v = x[j] - mean;
         sum += v*v;
         sum3 += v*v*v;
      }
      double m2 = sum / n;
      double m3 = sum3 / n;

      v = m3 / Math.pow (m2, 1.5);
      Function f = new Function (v);
      double t0 = 0;
      double tlim = Math.cbrt (v);
      if (tlim <= 0) {
         t0 = tlim;
         tlim = 10;
      }
      double t = RootFinder.brentDekker (t0, tlim, f, 1e-5);
      if (t <= 0)
         throw new UnsupportedOperationException("t <= 0;   no MLE");
      double xi = mean - Math.sqrt(m2 / t);
      if (xi >= xmin)
         // throw new UnsupportedOperationException("xi >= xmin;   no MLE");
         xi = xmin - 1.0e-1;
      v = 1 + m2 / ((mean - xi)*(mean - xi));
      double delta = 1.0 / Math.sqrt((Math.log(v)));
      v = Math.sqrt(v);
      double lambda = (mean - xi) / v;
      double[] param = new double[3];
      param[0] = delta;
      param[1] = xi;
      param[2] = lambda;
      return param;
   }


   private static class Optim implements Uncmin_methods
   {
      // minimizes the loglikelihood function
      private int n;
      private double[] X;
      private double xmin;      // min{X_j}
      private static final double BARRIER = 1.0e100;

      public Optim (double[] X, int n, double xmin) {
         this.n = n;
         this.X = X;
         this.xmin = xmin;
      }

      public double f_to_minimize (double[] par) {
         // par = [0, delta, xi, lambda]
         // arrays in Uncmin starts at index 1; par[0] is unused
         double delta  = par[1];
         double xi     = par[2];
         double lambda = par[3];
         if (delta <= 0.0 || lambda <= 0.0)         // barrier at 0
            return BARRIER;
         if (xi >= xmin)
            return BARRIER;

         double loglam = Math.log(lambda);
         double v, z;
         double sumv = 0.0;
         double sumz = 0.0;
         for (int j = 0; j < n; j++) {
            v = Math.log (X[j] - xi);
            z = delta * (v - loglam);
            sumv += v;
            sumz += z*z;
         }

         return sumv + sumz / 2.0 - n *Math.log(delta);
      }

      public void gradient (double[] x, double[] g)
      {
      }

      public void hessian (double[] x, double[][] h)
      {
      }
   }
\end{hide}
\end{code}
%%%%%%%%%%%%%%%%%%%%%%%%%%%%%%%%%%%%%%%%%
\subsubsection* {Constructors}

\begin{code}

   public JohnsonSLDist (double gamma, double delta)\begin{hide} {
      this (gamma, delta, 0, 1);
   }\end{hide}
\end{code}
  \begin{tabb}
    Same as \method{JohnsonSLDist}{double,double,double,double}
    {\texttt{(gamma, delta, 0, 1)}}.
  \end{tabb}
\begin{code}

   public JohnsonSLDist (double gamma, double delta,
                         double xi, double lambda)\begin{hide} {
      super (gamma, delta, xi, lambda);
      setLastParams(xi);
   }\end{hide}
\end{code}
  \begin{tabb} Constructs a \texttt{JohnsonSLDist} object
   with shape parameters $\gamma$ and $\delta$,
   location parameter $\xi$, and scale parameter $\lambda$.
  \end{tabb}

%%%%%%%%%%%%%%%%%%%%%%%%%%%%%%%%%%%
\subsubsection* {Methods}
\begin{hide}
\begin{code}

   private void setLastParams(double xi) {
      supportA = xi;
   }

   public double density (double x) {
      return density (gamma, delta, xi, lambda, x);
   }

   public double cdf (double x) {
      return cdf (gamma, delta, xi, lambda, x);
   }

   public double barF (double x) {
      return barF (gamma, delta, xi, lambda, x);
   }

   public double inverseF (double u){
      return inverseF (gamma, delta, xi, lambda, u);
   }

   public double getMean() {
      return JohnsonSLDist.getMean (gamma, delta, xi, lambda);
   }

   public double getVariance() {
      return JohnsonSLDist.getVariance (gamma, delta, xi, lambda);
   }

   public double getStandardDeviation() {
      return JohnsonSLDist.getStandardDeviation (gamma, delta, xi, lambda);
   }
\end{code}
\end{hide}\begin{code}

   public static double density (double gamma, double delta,
                                 double xi, double lambda, double x)\begin{hide} {
      if (lambda <= 0)
         throw new IllegalArgumentException ("lambda <= 0");
      if (delta <= 0)
         throw new IllegalArgumentException ("delta <= 0");

      if (x <= xi)
         return 0;
      double y = (x - xi)/lambda;
      double z = gamma + delta*Math.log (y);
      if (z >= 1.0e10)
         return 0;
      return delta * Math.exp (-z*z/2.0)/(lambda*y*Math.sqrt (2.0*Math.PI));
   }\end{hide}
\end{code}
\begin{tabb} Returns the density function $f(x)$.
\end{tabb}
\begin{code}

   public static double cdf (double gamma, double delta,
                             double xi, double lambda, double x)\begin{hide} {
      if (lambda <= 0)
         throw new IllegalArgumentException ("lambda <= 0");
      if (delta <= 0)
         throw new IllegalArgumentException ("delta <= 0");

      if (x <= xi)
         return 0.0;
      double y = (x - xi)/lambda;
      double z = gamma + delta*Math.log (y);
      return NormalDist.cdf01 (z);
   }\end{hide}
\end{code}
 \begin{tabb}
  Returns the  distribution function $F(x)$.
 \end{tabb}
\begin{code}

   public static double barF (double gamma, double delta,
                              double xi, double lambda, double x)\begin{hide} {
      if (lambda <= 0)
         throw new IllegalArgumentException ("lambda <= 0");
      if (delta <= 0)
         throw new IllegalArgumentException ("delta <= 0");

      if (x <= xi)
         return 1.0;
      double y = (x - xi)/lambda;
      double z = gamma + delta*Math.log (y);
      return NormalDist.barF01 (z);
   }\end{hide}
\end{code}
  \begin{tabb}
  Returns the complementary distribution function $1-F(x)$.
 \end{tabb}
\begin{code}

   public static double inverseF (double gamma, double delta,
                                  double xi, double lambda, double u)\begin{hide} {
      if (lambda <= 0)
         throw new IllegalArgumentException ("lambda <= 0");
      if (delta <= 0)
         throw new IllegalArgumentException ("delta <= 0");
      if (u > 1.0 || u < 0.0)
          throw new IllegalArgumentException ("u not in [0,1]");

      if (u >= 1.0)
         return Double.POSITIVE_INFINITY;
      if (u <= 0.0)
         return xi;

      double z = NormalDist.inverseF01 (u);
      double t = (z - gamma)/delta;
      if (t >= Num.DBL_MAX_EXP*Num.LN2)
         return Double.POSITIVE_INFINITY;
      if (t <= Num.DBL_MIN_EXP*Num.LN2)
         return xi;

      return xi + lambda * Math.exp(t);
   }\end{hide}
\end{code}
  \begin{tabb}
  Returns the inverse distribution function $F^{-1}(u)$.
 \end{tabb}
\begin{code}

   public static double[] getMLE (double[] x, int n)\begin{hide} {
      if (n <= 0)
         throw new IllegalArgumentException ("n <= 0");

      int j;
      double xmin = Double.MAX_VALUE;
      for (j = 0; j < n; j++) {
         if (x[j] < xmin)
            xmin = x[j];
      }
      double[] paramIn = new double[3];
      paramIn = initPar(x, n, xmin);
      double[] paramOpt = new double[4];
      for (int i = 0; i < 3; i++)
         paramOpt[i+1] = paramIn[i];

      Optim system = new Optim (x, n, xmin);

      double[] xpls = new double[4];
      double[] fpls = new double[4];
      double[] gpls = new double[4];
      int[] itrcmd = new int[2];
      double[][] a = new double[4][4];
      double[] udiag = new double[4];

      Uncmin_f77.optif0_f77 (3, paramOpt, system, xpls, fpls, gpls,
                             itrcmd, a, udiag);

      double[] param = new double[4];
      param[0] = 0;
      for (int i = 1; i <= 3; i++)
         param[i] = xpls[i];
      return param;
   }\end{hide}
\end{code}
\begin{tabb}
   Estimates the parameters $(\gamma$, $\delta$, $\xi$, $\lambda)$ of the
   \emph{Johnson $S_L$} distribution using the maximum likelihood method,
   from the $n$ observations $x[i]$, $i = 0, 1,\ldots, n-1$.
   The estimates are returned in a 4-element array in the order
   [0, $\delta$, $\xi$, $\lambda$] (with $\gamma$ always set to 0).
\end{tabb}
\begin{htmlonly}
   \param{x}{the list of observations to use to evaluate parameters}
   \param{n}{the number of observations to use to evaluate parameters}
   \return{returns the parameters [0, $\delta$, $\xi$, $\lambda$]}
\end{htmlonly}
\begin{code}

   public static JohnsonSLDist getInstanceFromMLE (double[] x, int n)\begin{hide} {
      double param[] = getMLE (x, n);
      return new JohnsonSLDist (0, param[0], param[1], param[2]);
   }\end{hide}
\end{code}
\begin{tabb}
   Creates a new instance of a \emph{Johnson $S_L$} distribution with
   parameters 0, $\delta$, $\xi$ and
   $\lambda$ over the interval $[\xi,\infty]$ estimated using the maximum likelihood
    method based on the $n$ observations $x[i]$, $i = 0, 1, \ldots, n-1$.
\end{tabb}
\begin{htmlonly}
   \param{x}{the list of observations to use to evaluate parameters}
   \param{n}{the number of observations to use to evaluate parameters}
\end{htmlonly}
\begin{code}

   public static double getMean (double gamma, double delta,
                                 double xi, double lambda)\begin{hide} {
      if (lambda <= 0.0)
         throw new IllegalArgumentException ("lambda <= 0");
      if (delta <= 0.0)
         throw new IllegalArgumentException ("delta <= 0");

      double t = 1.0 / (2.0 * delta * delta) - gamma / delta;
      return (xi + lambda * Math.exp(t));
   }\end{hide}
\end{code}
\begin{tabb}  Returns the mean
\begin{latexonly}
   \[E[X] = \xi + \lambda e^{1/2\delta^2 - \gamma/\delta}\]
\end{latexonly}%
   of the Johnson $S_L$ distribution with parameters $\gamma$, $\delta$, $\xi$
   and $\lambda$.
\end{tabb}
\begin{htmlonly}
   \return{the mean of the Johnson $S_L$ distribution
    $E[X] = \xi + \lambda e^{1/2\delta^2 - \gamma/\delta}$}
\end{htmlonly}
\begin{code}

   public static double getVariance (double gamma, double delta,
                                     double xi, double lambda)\begin{hide} {
      if (lambda <= 0.0)
         throw new IllegalArgumentException ("lambda <= 0");
      if (delta <= 0.0)
         throw new IllegalArgumentException ("delta <= 0");

      double t = 1.0 / (delta * delta) - 2 * gamma / delta;
      return lambda * lambda * Math.exp(t) * (Math.exp(1.0 / (delta * delta)) - 1);
   }\end{hide}
\end{code}
\begin{tabb}  Returns the variance
\begin{latexonly}
    \[\mbox{Var}[X] = \lambda^2 \left(e^{1/\delta^2} - 1\right)
      e^{1/\delta^2 - 2\gamma/\delta}\]
\end{latexonly}%
   of the Johnson $S_L$ distribution with parameters $\gamma$, $\delta$, $\xi$
   and $\lambda$.
\end{tabb}
\begin{htmlonly}
   \return{the variance of the Johnson $S_L$ distribution
    $\mbox{Var}[X] = \lambda^2 \left(e^{1/\delta^2} - 1\right)
      e^{1/\delta^2 - 2\gamma/\delta}$}
\end{htmlonly}
\begin{code}

   public static double getStandardDeviation (double gamma, double delta,
                                              double xi, double lambda)\begin{hide} {
      return Math.sqrt (JohnsonSLDist.getVariance (gamma, delta, xi, lambda));
   }\end{hide}
\end{code}
\begin{tabb}  Returns the standard deviation of the Johnson $S_L$
   distribution with parameters $\gamma$, $\delta$, $\xi$, $\lambda$.
\end{tabb}
\begin{htmlonly}
   \return{the standard deviation of the Johnson $S_L$ distribution}
\end{htmlonly}
\begin{code}

   public void setParams (double gamma, double delta,
                          double xi, double lambda)\begin{hide} {
      setParams0(gamma, delta, xi, lambda);
      setLastParams(xi);
   }\end{hide}
\end{code}
  \begin{tabb}
  Sets the value of the parameters $\gamma$, $\delta$, $\xi$ and
  $\lambda$ for this object.
 \end{tabb}

\begin{code}\begin{hide}
}\end{hide}
\end{code}
